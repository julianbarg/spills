\documentclass[12pt, man, natbib]{apa6}
\usepackage[USenglish]{babel}
\usepackage{setspace}
\usepackage{hyperref}

\title{Pipeline Spills \& Learning}
\shorttitle{Pipeline Spills \& Learning}
\author{Julian Barg\\barg.julian@gmail.com}
\affiliation{Ivey Business School}
\setcitestyle{authoryear, open={()},close={)},citesep={,},aysep=}


% \abstract{}


\begin{document}
	
	\maketitle
	
	\singlespacing
	
	\section{}
	
	To get a qualitative insight into organizational learning in the pipeline industry, one central piece is to look at incidents and operators' responses. Fortunately, the prominent pipeline incidents are relatively well publicised, if not in the media then at least in the work of the National Transportation Safety Board (NTSB). 
	
	Pipeline incidents get a varying degree of coverage. Some events receive almost no attention, like a 1992 gas explosion in Wesley, near Brenham, Texas, where three died. Factiva lists only 12 hits for the keywords "Brenham" and "explosion". Other spills attract significantly more attention, such as a 1989 pipeline explosion in San Bernadino, California that briefly made national news, from the New York Times to the Los Angeles Times.\footnote{See https://www.nytimes.com./1989/05/26/us/california-city-s-2d-disaster-in-2-weeks.html, accessed 2020-04-22, and https://www.latimes.com/archives/la-xpm-1989-05-25-mn-1056-story.html, accessed 2020-04-22.} The explosion occurred two weeks after a train had derailed in the same location. The operator had failed to detect that the pipeline had been damaged during the clean-up work of the train incident.

\bibliography{bibliography}

\end{document}