\documentclass[12pt, man, natbib]{apa6}
\usepackage[USenglish]{babel}
\usepackage{setspace}
\usepackage{hyperref}

\title{Pipeline Spills \& Learning}
\shorttitle{Pipeline Spills \& Learning}
\author{Julian Barg\\barg.julian@gmail.com}
\affiliation{Ivey Business School}
\setcitestyle{authoryear, open={()},close={)},citesep={,},aysep=}


% \abstract{}


\begin{document}
	
	\maketitle
	
	\singlespacing
	
	\section{}
	
	To get a qualitative insight into organizational learning in the pipeline industry, one central piece is to look at incidents and operators' responses. Fortunately, the prominent pipeline incidents are relatively well publicised, if not in the media then at least in the work of the National Transportation Safety Board (NTSB). 
	
	Pipeline incidents get a varying degree of coverage. Some events receive almost no attention, like a 1992 gas explosion in Wesley, near Brenham, Texas, where three died. Factiva lists only 12 hits from local newspapers for the keywords "Brenham" and "explosion". Other spills attract significantly more attention, such as a 1989 pipeline explosion in San Bernadino, California that briefly made national news, from the New York Times to the Los Angeles Times.\footnote{See https://www.nytimes.com./1989/05/26/us/california-city-s-2d-disaster-in-2-weeks.html, accessed 2020-04-22, and https://www.latimes.com/archives/la-xpm-1989-05-25-mn-1056-story.html, accessed 2020-04-22.} The explosion occurred two weeks after a train had derailed in the same location. The operator had failed to uncover that the pipeline had been damaged during the clean-up work of the train incident. We are missing some data that would be available for more recent incidents, but interestingly, based on the number of injuries and fatalities, the two incidents were quite similar.
	
	Fortunately, both incidents are covered equally well by the NTSB. The NTSB is called on to investigate airplane and other transportation-related incidents, including pipeline incidents. The NTSB aims to be on the site of the incident "as quickly as possible"\footnote{https://www.ntsb.gov/investigations/process/Pages/default.aspx, accessed 2020-04-23.}, rather than collecting data after the fact. The selection process of the NTSB is not entirely clear, but they are often called on to investigate incidents of state-wide and national importance, where policy makers and the public ask the question of "how could this happen". In recent years, they have published between 8 (2019) and 1 (2016) reports. Many if not all of the historically relevant incidents were covered by the NTSB.

\bibliography{bibliography}

\end{document}