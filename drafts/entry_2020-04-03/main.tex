\documentclass[12pt, man, natbib]{apa6}
\usepackage[USenglish]{babel}
\usepackage{setspace}
\usepackage{xcolor}
\usepackage{soul}

\title{Positioning the Phenomenon of Pipeline Spills in the Literature}
\shorttitle{Positioning}
\author{Julian Barg\\barg.julian@gmail.com}
\affiliation{Ivey Business School}
\setcitestyle{authoryear, open={()},close={)},citesep={,},aysep=}


% \abstract{}


\begin{document}
	
	\maketitle
	
	\singlespacing
	
	\section{}	
	
	The phenomenon of pipeline spills can be approached from different angles. Apart from learning, there is the angle of other organizations' responses to a focal firm's oil spill (especially from a network perspective). And there is a cross-sectional puzzle: what differentiates a pipeline operator that experiences many spills from one that experiences few? The learning perspective encompasses a longitudinal view: when or how does a pipeline operator improve its track record? 
	
	This last longitudinal story, which the learning angle encompasses, is attractive with regard to the motivation of such piece. We already know that the environmental impacts exist, and we have a hunch that the response, save for a couple major incidents, is inadequate (since the responding actors have not been successful in ending the phenomenon of pipeline spills). If we were to take the current status quo as a starting point, we might end up with some implications that can be applied to the pipeline operator industry as it currently stands, implications that could yield some concrete medium term improvements over the current status quo\footnote{Although I am afraid we might just end up with an unimaginative statement along the lines of "Ey, we need enforcement, jo!"}.
	
	However, the learning literature is not a homogeneous entity. There are (at least) three different avenues from learning into pipeline safety, and eventually into environmental performance: (1) rare events, (2) organizational behavior, and (3) population-level learning. Very few oil spills have a disproportionate (relative to the median) spill volume; when those spills occur near a water way, the oil become difficult to recover, and the environmental impact can be enormous. For instance, on Dec. 5, 2016...

\bibliography{bibliography}

\end{document}