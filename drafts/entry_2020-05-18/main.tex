\documentclass[12pt, man, natbib]{apa6}
\usepackage[USenglish]{babel}
\usepackage{setspace}
\usepackage{hyperref}

\title{Better, faster, cleaner?}
\shorttitle{Better, faster, cleaner?}
\author{Julian Barg\\barg.julian@gmail.com}
\affiliation{Ivey Business School}
\setcitestyle{authoryear, open={()},close={)},citesep={,},aysep=}


% \abstract{}


\begin{document}
	
	\maketitle
	
	\singlespacing
	
	\section{}
	How do we think of the modern world? Do we think of a world of energy that comes from the sun, pocket-sized devices that allow us to instantaneously communicate with anybody in the world, and rockets that reenter the atmosphere and land by themselves? Or do we think of crumbling infrastructure, of a health system that could be overcome by an epidemic at any time, and supply chains that still rely on third world labor under poor conditions? The fact of the matter is that both descriptions are accurate. In the context of pipelines, we see a contradictory reality of pipes that are literally rotting in the ground, and modern Supervisory Control And Data Acquisition (SCADA) as well as Leak Detection Systems (LDS). Still commonplace are large oil spills that pollute the environment, lead to fatalities, and make residents sick [Show data? Would probably put that in appendix.]. The standards to which the pipeline network has been modernized varies significantly between operators. Organizational learning in this arena is determined by intentionality.
	
	Pipeline spills have many different dimensions. (1) How much of the commodity is spilled? How much of it was recovered or (2) what was the net volume spilled? Were people injured or did somebody even die? How many (3) injuries or (4) fatalities were there? How much (5) damage did a fire or explosion cause, for instance to residential housing? Below are some insights on learning from the most severe spills in these categories.
	
	Three perceptual dimensions determine whether an event or outcome yields insights that could be "pathbreaking"--in our context meaning that these insights could prevent future oil spills. These three dimensions are not entirely distinct, rather, they have some overlap or correlation. With regards to an oil spill, the three dimensions are (1) the perceived \textit{severity} of a spill \citep[whether the spill is recognized as a "failure",][]{Madsen2010}, (2) the perceived \textit{complexity} of a spill, and (3) the ease of \textit{assigning responsibility}. In the general case, the first dimension could be reframed as divergence from aspiration. In the following some examples of these dimensions playing out in the context of oil spills.
	
	Spill A occurred 20002 in Frost, Texas. The spill was neither recognized as severe, nor as complex, and responsibility was not explored (at least in public-facing documents). In terms of the net spill volume, this spill was the largest onshore spill on record, with 33,010 barrels of HVLs spilled. The spill had no consequences, as under normal conditions, HVLs dissipate quickly. An explosion and a fire occurred, but because the area is quite remote, there were no injuries and no damage to property. The firm was not held responsible for damages to wildlife or vegetation, or any effect the escaping HVLs might have had on the climate. Damages were incurred in the form of lost HVLs, however, from a company perspective the cost of these lost HVLs might be offset by the savings achieved by keeping maintenance of the 1966 pipeline to a minimum. Chevron's report on the spill reports the cause as corrosion.
	
	In contrast Spill A, Spill B had severe consequences. Spill B occured 1992 in Brenham, Texas. While initially, a pipeline in the area was believed to be leaking, eventually an investigation by the National Transportation Safety Board (NTSB) uncovered at least eight important contributing factors, and not one single root cause. Legal responsibility could not be easily assigned, as fatal mistakes happened on different organizational levels. 
	
	The investigation into Spill B occurred because of the \textit{severity} of the spill [could reference mission of NTSB here]. Subsequently, the investigation uncovered some of the underlying issues regarding Spill B, for instance with regard to server maintenance. Only through the investigation was the full \textit{complexity} of Spill B uncovered. Because Spill B, in its full complexity, includes issues with server maintenance, a potential learning opportunity for the industry with regard to server maintenance emerged. Spill A of course did not yield any similarly important insights. If at the time, the spill had been investigated further, it would have probably emerged that corrosion was not the only reason for the spill; rather, an investigation would have focused on \textit{why} corrosion occurred and was not identified through periodic examination of the line. But because the spill was not sever and did not warrant a further investigation (or any outside attention really), potential causes for Spill A were not uncovered, and the notion of low \textit{complexity} went unchecked.
	
	Spill C occurred in Walnut Creek, California in 2004. The spill was \textit{severe}--5 contractors died on site. But the question of \textit{responsibility} for this spill was also pretty clear-cut. Construction workers who were digging a trench for a water pipe hit the oil pipeline with a backhoe. Incident C did not garner the same kind of national attention that Spill A did, but still enough to warrant an investigation by the California Fire Marshall. The report by the Fire Marshall uncovered that Kinder Morgan, the operator, made at least four mistakes leading up to the mistake. Kinder Morgan was eventually fined \$15 million on six felony counts and one year later entered an agreement with the federal government to invest another \$90 million in safety measures after a series of incidents, including Spill C. The \textit{severity} of the incident led to the underlying \textit{complexity} being uncovered. Had this incident occurred under slightly different circumstances (no ignition of the oil by a nearby blowtorch, different state, less paper trail) it would be thinkable that less of the complexity were uncovered.\footnote{Unfortunately, many, including me, would probably read the headline "Construction workers hit gas pipeline, 5 dead" and not bat an eye. Fortunately, in this case it was uncovered that the workers were not solely at fault.}	
	
	For learning, the dynamic of severity, complexity and responsibility means that the number of lessons learned is not solely dependent on whether an incident or event is objectively a failure. The learning process is non-linear, in that incidents or events that garner attention can lead to surprising learning insights. For instance, Spill B uncovered problems regarding SCADA systems and the servers that power them, although the spill might have still occurred if the SCADA system worked perfectly. A limitation to the learning process is the nature of incidents or events that have already occurred and that we understand the causality of, which allows for intentionally addressing that issue.\footnote{To raise an example of causality: Incidents that lead to direct fatalities, such as pipeline explosions, are more likely to lead to learning than those that lead to indirect deaths, such as the release of carcinogenic substances.}
	
	A good example of intentionality is Spill D, a massive spill that occurred in Houston, Texas in 1994. Hurricane Rosa led to at least 4 pipelines under the San Jacinto River spilling their contents in to the river. Flames, at one point a hundred feet high, engulfed the river. Over 500 mostly smoke-related occurred. The spill was undeniable evidence that the pipelines were not buried deep enough under the river. 
	
	
	
%	
%	But the perceived complexity, and whether this complexity is investigated has an impact on lessons learned. No doubts the cause of the second incident is more complex than the official narrative lets on. But because [perceived as simple or success], certainly less learning has taken place, as evident from the lack of analysis. [although, what has happened internally?]
	

\bibliography{bibliography}

\end{document}