\documentclass[12pt, man, natbib]{apa6}
\usepackage[USenglish]{babel}
\usepackage{setspace}
\usepackage{hyperref}

\title{Better, faster, cleaner?}
\shorttitle{Better, faster, cleaner?}
\author{Julian Barg\\barg.julian@gmail.com}
\affiliation{Ivey Business School}
\setcitestyle{authoryear, open={()},close={)},citesep={,},aysep=}


% \abstract{}


\begin{document}
	
	\maketitle
	
	\singlespacing
	
	\section{}
	How do we think of the modern world? Do we think of a world of energy that comes from the sun, pocket-sized devices that allow us to instantaneously communicate with anybody in the world, and rockets that reenter the atmosphere and land by themselves? Or do we think of crumbling infrastructure, of a health system that could be overcome by an epidemic at any time, and supply chains that still rely on third world labor under poor conditions? The fact of the matter is that both descriptions are accurate. In the context of pipelines, we see a contradictory reality of pipes that are literally rotting in the ground, and modern Supervisory Control And Data Acquisition (SCADA) as well as Leak Detection Systems (LDS). Still commonplace are large oil spills that pollute the environment, lead to fatalities, and make residents sick [Show data? Would probably put that in appendix.]. The standards to which the pipeline network has been modernized varies significantly between operators. Organizational learning in this arena is determined by intentionality.
	
	Pipeline spills have many different dimensions. (1) How much of the commodity is spilled? How much of it was recovered or (2) what was the net volume spilled? Were people injured or did somebody even die? How many (3) injuries or (4) fatalities were there? How much (5) damage did a fire or explosion cause, for instance to residential housing? Below are some insights on learning from the most severe spills in these categories.
	
	Three perceptual dimensions determine whether an event or outcome yields insights that could be "pathbreaking"--in our context meaning that these insights could prevent future oil spills. These three dimensions are not entirely distinct, rather, they have some overlap or correlation. With regards to an oil spill, the three dimensions are (1) the perceived severity of a spill \citep[whether the spill is recognized as a "failure",][]{Madsen2010}, (2) the perceived complexity of a spill, and (3) the ease of assigning responsibility. In the general case, the first dimension could be reframed as divergence from aspiration. In the following some examples of these dimensions playing out in the context of oil spills.
	
	In Frost, Texas occured in 2002 a spill that was recognized as neither severe, nor complex, and responsibility was not explored (at least in public-facing documents). In terms of the net spill volume, this spill was the largest onshore spill on record, with 33,010 barrels of HVLs spilled. The spill had no consequences, as under normal conditions, HVLs dissipate quickly. An explosion and a fire occurred, but because the area is quite remote, there were no injuries and no damage to property. The firm was not held responsible for damages to wildlife or vegatation, or any effect the escaping HVLs might have had on the climate. Damages were incurred in the form of lost HVLs, however, from a company perspective the cost of these lost HVLs might be offset by the savings achieved by keeping maintenance of the 1966 pipeline to a minimum. Chevron's report on the spill reports the cause as corrosion.
	
	In contrast to the 2002 spill in Frost, a smaller scale 1992 spill in Brenham, Texas had severe consequences. 
	
%	Chemicals spilled might enter the human food chain and lead to a local spike of the cancer rate that is hard to trace back.	
%	
%	The incidents vary a lot in complexity. In some cases, one could almost speak of complex systems that cause a spill. In other cases, the cause of the incident is seemingly simple. For instance, a chain of at least 8 distinct technical and management failures led to a gas explosion in Brenham, Texas in 1992, which killed three. On the other hand, a large 2002 HVL spill of 33,010 barrels in Frost, Texas is traced back simply to a corroded pipe. This perceived complexity has an impact on the learning outcome. 
%		
%	Is that really true though? Corrosion only?
%	
%	But the perceived complexity, and whether this complexity is investigated has an impact on lessons learned. No doubts the cause of the second incident is more complex than the official narrative lets on. But because [perceived as simple or success], certainly less learning has taken place, as evident from the lack of analysis. [although, what has happened internally?]
	

\bibliography{bibliography}

\end{document}