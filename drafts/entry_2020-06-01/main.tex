\documentclass[12pt, man, natbib]{apa6}
\usepackage[USenglish]{babel}
\usepackage{setspace}
\usepackage{hyperref}

\title{A Full Picture of Organizational Learning}
\shorttitle{A Full Picture of Organizational Learning}
\author{Julian Barg\\barg.julian@gmail.com}
\affiliation{Ivey Business School}
\setcitestyle{authoryear, open={()},close={)},citesep={,},aysep=}


\abstract{}


\begin{document}
	
	\maketitle
	
	\singlespacing
	
	\section{}	
	Organizational learning has to address a complex reality. Organizations--mostly large international corporations--have given us pocket-sized devices that allow us to instantaneously communicate with anybody in the world face-to-face, we are efficiently harnessing the power of the sun to meet our energy demands, and there are rockets that reenter the atmosphere and land by themselves. At the same time there is crumbling infrastructure, a health system that could easily be overcome by an epidemic at any time, and supply chains that still rely on exploitation of third world labor and poor working conditions. The utopian vision of a future that holds both awe-inspiring technology and equity between people has not held true. Instead, we witness a coexistence of both utopian and dystopian elements of previous generations' imaginations of the future. Unless one defines exploitation and short-term thinking as know-how to be acquired, knowledge has proven not to be cumulative. The organizational learning literature should aim to explain this \textit{drift}, rather than \textit{accumulation}, of knowledge. Organizations learn within the confines of their choices, and in other areas knowledge may be lost.
	
	An important mechanism are organizations' choices. How or why do organizations choose to learn something?\footnote{We do not mean to imply that organizations somehow make conscious decisions to learn something. Rather we want to illustrate that there are many paths for organizations to take, and it is not as straightforward as organizations learning or not learning.} Organizational learning is more likely to occur in an area when there is purpose to or a motivation for it. When there is a positive feedback to an effort, organizations are more likely to pursue it further. Organizations are also hard-pressed to improve something that they cannot measure, usually in terms of hard numbers.\footnote{hence the extensive discussion in the learning literature on rare events, near-failures, and near-misses}
	
	One of the most prolific streams of the learning literature until the 1990s been the discourse on learning curves. When plotting out input and output, one can observe that factories improve productivity over time. The same learning effect has been demonstrated for a selected number of other important metrics \citep{Argote2013a}. Are there things that organizations cannot learn? The literature on organizational learning does not give a direct answer to that question--where there is no learning, there is nothing to report in this stream of research. The behavioral stream on organizational research suggests three mechanisms: (1) organizations do not know what they cannot measure; (2) if they are able to measure it, but not to improve it, they may not bother to measure it; and (3) if they are capable of measuring it and improving it, we still might not hear about it if they do not do so.
	
	By addressing this topic, we hope to initiate a discourse on some of the basic assumptions of the learning literature. The learning literature has made great strides since scholars first created a formal theory \citep{March1963}. We now have models available of knowledge generation, transfer, and population level learning. The examples in the opening paragraph indicate that there are also some limits to organizational learning. The behavioral theory of organizational learning can provide some general insights on limitations to organizational learning, what is being learned, and what larger trends we can expect.
	
	\section{Why organizational learning matters}
	
	Organizational learning originated in two separate streams of literature. [(1) stuff about learning curves and \textit{knowledge-based} learning].
	
	(2) \textit{Behavioral learning} makes its first appearance in \citet{March1963}. The topic is frequently revisited by March thereafter [citations]. The original work 1963 contributes to the research stream on the theory of the firm, and as such it is not entirely descriptive but provides many assumptions, too. Specifically, the work was motivated by the observation that individuals are boundedly rational, and therefore organizations would be \textit{adaptively rational} systems.\footnote{And not \textit{omnisciently rational} systems, as implied by conventional economic theory at the time.}
	
	Later, the debate has given way to studies of mechanisms and microfoundations of learning. Yet, the theoretical foundations of the stream still speak to organizational learning. Reintroducing a few basic concepts, such as bounded rationality, departments, and goals, in addition to more recent ones, such as population level learning, failures, and near-failures, allows us to explore some of the limitations of learning. Those limits to organizational learning cannot be captured by analyses that only study how organizational learning occurs. To study these limitations would provide us with a more realistic model of organizational learning, as it would describe reality more accurately. 
	
	Current research on organizational learning is motivated by the question of what explains variation in learning \citep[p. 2]{Argote2013a}, which is not that far off of our research question of how or why organizations choose to learn something. The difference might seem like semantics, but there is an important difference built into the two statements. Organizations explore few of the pathways that are available to them. The existing literature on learning curves\footnote{Although it admittedly is a little bit dated by now.} demonstrates the difference well: the dependent variables in the research articles are typically metrics that are obvious for organizations to pursue. The variation in progress toward improving those metrics then is only limited by organizations' capabilities to learn. When we emphasize organizations' choices instead, we emphasize that for every metric that an organization (and typically a population or industry as a whole), there is a large number of metrics that is not being pursued. For instance, these metrics may not be pursued because these metrics (1) have not relationship with the goals that organization have selected for themselves, (2) because they have a relationship with the goals of the organization, but the relationship is seen as weak, or (3) because the organization does not recognize the relationship with its goals at the time.
	
	An example for (1) would be that car manufacturers typically do not consider to engage in food production.\footnote{Not a conceived example, see \url{https://en.wikipedia.org/wiki/Volkswagen\_currywurst}.} (2) Organizations can typically reduce accident cost by investing in safety, but safety equipment is often seen as not yielding sufficient returns in cost reductions to warrant the investment. (3) For instance, industry incumbents such as Harley Davidson may only decide to innovate when hard-pressed by economic conditions, rather than innovating at any time to maximizing profit. Overall, the purpose of a theory of organizational learning along these lines of thinking would be to move the starting point: instead of asking why learning varies where it occurs, the question becomes why learning occus specifically in the few areas where it is being observed.
	
	\section{Organizational learning under conditions of bounded rationality}
	
%	Grab March's later work and describe how individual efforts translate into organizational learning.
	
	

%	The organizational learning literature was originally motivated by two separate observations. (1) Cyert and March explored how under conditions of bounded rationality, organizations could adapt to the environment. They concluded that firms routinely make small, predetermined changes to 
%	
%	Limits to adaption by March
%	
%	March observed that the theory of rationality [what would be a more precise term?] did not hold, and instead turned to bounded rationality. His initial conceptualization of organizational learning was an attempt to reconcile bounded rationality and organizational intelligence \citep{March1975}. (2) The second observation was made by \citet{Wright1936}--he noticed that with increasing production quantities, production cost would fall, and approach a specific value. Later, this discovery would come to be called the learning curve.
	
	

\bibliography{bibliography}

\end{document}