\documentclass[12pt, man, natbib]{apa6}
\usepackage[USenglish]{babel}
\usepackage{setspace}
\usepackage{hyperref}

\title{The boundedly rational organization: How to conceptualize organizational learning}
\shorttitle{the boundedly rational organization}
\author{Julian Barg\\barg.julian@gmail.com}
\affiliation{Ivey Business School}
\setcitestyle{authoryear, open={()},close={)},citesep={,},aysep=}


% \abstract{}


\begin{document}
	
	\maketitle
	
	\singlespacing
	
	\section{}	
	
	[Maybe add paragraph about coexistence of high-tech and backwards stuff here? To clarify what kind of world it is I am talking about.]
	
	How do organizations choose what to learn? Organizational learning is more likely to occur when there is a value or purpose to it. When an organization makes a new product, and this product does not sell well, the lesson learned may be "we have tried making this product, and there is no interest from the general public for this product" [could potentially change this example with something sustainability-related]. Organizations are also hard-pressed to improve something that they cannot measure, usually in terms of hard numbers \footnote{hence the extensive discussion in the learning literature on rare events, near-failures, and near-misses}.
	
	One of the most prolific streams of the learning literature until the 1990s been the discourse on learning curves. When plotting out input and output, one can observe that factories improve productivity over time. The same learning effect has been demonstrated for a selected number of other important metrics (Argote 2013). Are there things that organizations cannot learn? The literature on organizational learning does not answer that question--where there is no learning, there is nothing to report in this stream of research. The behavioral stream on organizational research suggests that organizations do not know what they cannot measure; if they are able to measure it, but not to improve it, they may not bother to measure it; and if they are capable of measuring it and improving it, we still might not hear about it if they do not do so.
	
	By addressing this topic, we hope to initiate a discourse on some of the basic assumptions of the learning literature. The learning literature has made great progress since scholars first started to theorize on the topic (Cyert and March 1963). 
	
%	Organizational learning (broadly speaking) was first proposed by Cyert and March for one purpose: to explain successful organizational adaption under the limitation of individual bounded rationality (Cyert and March 1993). Broadly speaking, March observed that organizations were not intelligent, but that they also did not behave completely irrational. Their behavior follows some rules, and in a changing environment, they are able to adjust. Individual cognitive efforts did not accumulate in an omniscient rational system, but they also were not in vain.
%	
%	The organizational learning literature was originally motivated by two separate observations. (1) Cyert and March explored how under conditions of bounded rationality, organizations could adapt to the environment. They concluded that firms routinely make small, predetermined changes to 
%	
%	Limits to adaption by March
%	
%	March observed that the theory of rationality [what would be a more precise term?] did not hold, and instead turned to bounded rationality. His initial conceptualization of organizational learning was an attempt to reconcile bounded rationality and organizational intelligence \citep{March1975}. (2) The second observation was made by \citet{Wright1936}--he noticed that with increasing production quantities, production cost would fall, and approach a specific value. Later, this discovery would come to be called the learning curve.
	
	

\bibliography{bibliography}

\end{document}