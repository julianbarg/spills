\documentclass[12pt, man, natbib]{apa6}
\usepackage[USenglish]{babel}
\usepackage{setspace}
\usepackage{hyperref}

\title{The behavioral approach to organizational learning}
\shorttitle{The behavioral approach to organizational learning}
\author{Julian Barg\\barg.julian@gmail.com}
\affiliation{Ivey Business School}
\setcitestyle{authoryear, open={()},close={)},citesep={,},aysep=}


\abstract{}


\begin{document}
	
	\maketitle
	
	\singlespacing
	
	\section{}	
	Organizational learning has to address a complex reality. Organizations--mostly large international corporations--have given us pocket-sized devices that allow us to instantaneously communicate with anybody in the world face-to-face, we are efficiently harnessing the power of the sun to meet our energy demands, and there are rockets that reenter the atmosphere and land by themselves. At the same time there is crumbling infrastructure, a health system that could easily be overcome by an epidemic at any time, and supply chains that still rely on exploitation of third world labor under poor conditions. The utopian vision of a future that holds both awe-inspiring technology and equity between people has not held true. Instead, we witness a coexistence of both utopian and dystopian elements of previous generations' imaginations of the future. Unless one defines exploitation and short-term thinking as know-how to be acquired, knowledge cannot be seen as cumulative. Organizations learn within the confines of their choices, and in other areas knowledge may be lost.
	
	How do organizations choose what to learn? Organizational learning is more likely to occur when there is purpose to or a motivation for it. When there is a positive feedback to an effort, organizations are more likely to pursue it further. Organizations are also hard-pressed to improve something that they cannot measure, usually in terms of hard numbers.\footnote{hence the extensive discussion in the learning literature on rare events, near-failures, and near-misses}
	
	One of the most prolific streams of the learning literature until the 1990s been the discourse on learning curves. When plotting out input and output, one can observe that factories improve productivity over time. The same learning effect has been demonstrated for a selected number of other important metrics (Argote 2013). Are there things that organizations cannot learn? The literature on organizational learning does not give a direct answer to that question--where there is no learning, there is nothing to report in this stream of research. The behavioral stream on organizational research suggests three mechanisms: (1) organizations do not know what they cannot measure; (2) if they are able to measure it, but not to improve it, they may not bother to measure it; and (3) if they are capable of measuring it and improving it, we still might not hear about it if they do not do so.
	
	By addressing this topic, we hope to initiate a discourse on some of the basic assumptions of the learning literature. The learning literature has made great strides since scholars first created a formal theory (Cyert and March 1963). We now have models available of knowledge generation, transfer, and population level learning. The examples in the opening paragraph indicate that there are also some limits to organizational learning. The behavioral theory of organizational learning can provide some general insights on limitations to organizational learning, what is being learned, and what larger trends we can expect.
	
	\section{Why organizational learning matters}
	
	Organizational learning originated in two separate streams of literature. [(1) stuff about learning curves and knowledge-based learning].
	
	(2) The stream about \textit{behavioral learning} makes its first appearance in Cyert and March (1963). The topic is frequently revisited by March thereafter [citations]. The original work 1963 contributes to the research stream on the theory of the firm, and as such it is not just descriptive, but comes with many assumptions, too. Specifically, the work was motivated by the observation that individuals are boundedly rational, and therefore organizations would be \textit{adaptively rational} systems. \footnote{And not \textit{omnisciently rational} systems, as implied by conventional economic theory at the time.}
	
	If organizations are made up of individuals, and individuals are boundedly rational,  
	
	Hence organizations are not [describe adaptively rational]. They are adapting by [describe learning]. The individual contributes by [describe later work].
	
	
	\section{Organizational learning under conditions of bounded rationality}
	
	

%	The organizational learning literature was originally motivated by two separate observations. (1) Cyert and March explored how under conditions of bounded rationality, organizations could adapt to the environment. They concluded that firms routinely make small, predetermined changes to 
%	
%	Limits to adaption by March
%	
%	March observed that the theory of rationality [what would be a more precise term?] did not hold, and instead turned to bounded rationality. His initial conceptualization of organizational learning was an attempt to reconcile bounded rationality and organizational intelligence \citep{March1975}. (2) The second observation was made by \citet{Wright1936}--he noticed that with increasing production quantities, production cost would fall, and approach a specific value. Later, this discovery would come to be called the learning curve.
	
	

\bibliography{bibliography}

\end{document}