\documentclass[12pt, man, natbib]{apa6}
\usepackage[USenglish]{babel}
\usepackage{setspace}
\usepackage{hyperref}

\title{Power and Imagination: Takeaway from Foucault}
\shorttitle{Power and Imagination}
\author{Julian Barg\\barg.julian@gmail.com}
\affiliation{Ivey Business School}
\setcitestyle{authoryear, open={()},close={)},citesep={,},aysep=}


% \abstract{}


\begin{document}
	
	\maketitle
	
	\singlespacing
	
	\section{}
	A word without oil--would that be a utopia or a dystopia? For some, oil is a relic of the past, something that has to be overcome, ideally to be replaced with local, clean, renewable sources of energy. For others, oil is a reliable source of energy, abundantly available, maybe not entirely clean\footnote{Except maybe in the Troll-in-chief's "clean coal" universe.} but controllable--and a world without oil would be less stable, more prone to turmoil. Utopia or dystopia--the scenario of a world without oil seems equally unlikely from both perspectives.
	
	In the universe of the petroleum engineer, pipeline safety is a success story. Over the last 50 years, we have witnessed a complete automation of the area, starting with remote valves, onto more and more sophisticated Supervisory Control and Data Acquisition (\textit{SCADA}) systems, and finally \textit{smart pigs} that allow for the operator to evaluate the condition of the pipeline from the inside. Pipeline spills occur not because pipelines are inherently unsafe, but can usually be traced back to human error and/or a concatenation of unfortunate events.
	
	But despite these accomplishments, pipeline spills are still prevalent. Even prestige projects like the Keystone Pipeline do not live up to their promises. 

\bibliography{bibliography}

\end{document}