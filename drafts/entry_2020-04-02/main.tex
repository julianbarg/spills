\documentclass[12pt, man, natbib]{apa6}
\usepackage[USenglish]{babel}
\usepackage{setspace}
\usepackage{hyperref}

\title{Entry 2020-04-02 - The big picture}
\shorttitle{The big picture}
\author{Julian Barg\\barg.julian@gmail.com}
\affiliation{Ivey Business School}
\setcitestyle{authoryear, open={()},close={)},citesep={,},aysep=}


% \abstract{}


\begin{document}
	
	\maketitle
	
	\singlespacing
	
	\section{}
	One often encounters unreflected expectation of progress in all parts of this world. Even sustainability scholars, who, if asked, will probably have negative expectations about most companies' sustainability performance, can often be caught making unwarranted assumptions of progress. When the more unbiased view of the world would be to expect progress only where a mechanism can be identified.
	
	With this background in mind, let's turn to the phenomenon and theory at hand. Learning theory typically assumes that learning is a somewhat universal organizational phenomenon. Meaning it is assumed to apply (1) across organizations: "As organizations produce more of a product, the unit cost of production typically decreases at a decreasing rate" \citep[1]{Argote2013a}. And (2) across outcomes - "Research [...] had shown that outcomes in addition to the number of direct labor hours per unit followed a learning curve" \citep[6]{Argote2013a}.
		\footnote{The first might be generally true, at least for viable organizations. If some organizations "learn", and others do not, those that learn will be more likely to survive \citep{Cyert1963b}.}
	In line with these assumptions, organizational learning is quite generously defined as "change in the organization's knowledge that occurs as a function of experience" \citep[31]{Argote2013b}.
		
	In what cases would one not expect an organization to learn? There are two cases. Let's say an organization gathers experience on an arbitrary task, such as cleaning its facilities. If this arbitrary task does not have a major impact on organizational survival (which in most cases it is not), then the organization could get away with not learning in this area. That is the first case. Note that not learning in this case does not mean that the organization gets worse at the task. It simply means that the same resource input leads to the same outcome. Where the government or local stakeholders do not vigorously enforce environmental protection, activities with environmental impact can be irrelevant to environmental protection, and hence organizations could afford not to learn in these areas.
	
	Another relevant question is whether there are decision makers in the organization that believe that a function is important. An organization may learn in any direction, even if it is not relevant for survival, if there are advocates that push for learning in this area.
	
	\section{Side note: Organizational Learning vs. Resource Investment}
	
	As a side note, an extension of this view of deliberate learning would be to revisit the distinction between resource investment and learning. A lot of what is defined as organizational learning in the literature could actually be subsumed under resource investments. In particular, how can technology (acquisition or production of specialized equipment) be differentiated from capital investment? Relevant to organizational learning would only be that the organization now knows which tools to apply. Of course, this issue has been visited by others before me \citep[5]{Argote2013a}. But for pipelines it seems particularly relevant. Thicker pipelines, or pipelines made from the right materials, can withstand corrosion better, but it is more expensive. Regulations that require thicker pipelines do not constitute learning.

\bibliography{bibliography}

\end{document}
