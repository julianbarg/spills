\documentclass[12pt, man, natbib]{apa6}
\usepackage[USenglish]{babel}
\usepackage{setspace}

\title{Note 2019-12 on Interorganizational Learning}
\shorttitle{Note 2019-12}
\author{Julian Barg\\barg.julian@gmail.com}
\affiliation{Ivey Business School}
\setcitestyle{authoryear, open={()},close={)},citesep={,},aysep=}


% \abstract{}


\begin{document}
	
\maketitle

\singlespacing

\section{}	
	The learning literature includes works on learning curves, organizational knowledge, and of course organizational learning itself. This literature offers an extensive, and fairly tangible understanding of what knowledge is, how it is created and preserved, and how it can serve any one goal of an organization \citep{Argote2013}. It offers a bountiful supply of works that analyze the increases in performance, as we have seen them throughout most of the 20th century. This focus may be justified, as the term “learning” itself is associated with useful knowledge and implies in the very least a drift toward improvement. In addition, in the learning literature’s genes also slumbers a complex view of learning, choices, and change. In particular, \citet{Levitt1988} opened a new chapter by refocusing the literature on issues such as competency traps, superstitious learning, and the ambiguity of success .
	
	In March’s world of organizational learning, organizations exhibit complex behavioral traits, which go beyond a cumulative experience that comes with cumulative production. How would this view lend itself to a study of a learning phenomenon, and of mechanisms? To give an example, \citet{Madsen2010} in their hypothesis development pick up on the behavioral tradition, when they hypothesize that organizations which have experienced relatively few failures will misapply lessons learned from others’ successes and their own failure. The behavioral tradition is pushed aside however when the literature delivers clear-cut stories of performance-enhancing learning.
	
	The most important notion, that has partially been left behind, is that of goals. In A Behavioral Theory of the Firm, organizational goals receive the main attention \citep[see][]{Cyert1992-2}, and the theme is present in March’s later work as well \citep[especially][]{Levitt1988}. Competency traps describe a trade-off between long-run and short-run goals. Superstitious learning occurs because the relationship between routines, and outcomes (relative to goals) is ambiguous. And the ambiguity of success exists because “[g]oals are ambiguous and commitment to them is confounded by their relation to personal and subgroup objectives” \citep[p. 325]{Levitt1988}. 
	
	Again, Madsen and Desai \citet{Madsen2010} is a good starting point for as an example how these concepts (and thus the more complex, overlaying theory) could be given more life. They describe how organizations move the goalpost, when they redefine minor failures (or near-misses) as success, which in the real world eventually culminated in the Columbia space shuttle disaster \citep[p. 457]{Madsen2010}.
	
	The main challenge remains: organizational goals are a result of bargaining between different invidiuals or subcoalitions, with diverging preferences \citep{Cyert1992}. This issue should be kept in mind when we study learning. Cyert and March for instance describe potential conflicts between the very basic interests of production volume, inventory levels, sales, market share, and profitability \citep[pp. 46-49]{Cyert1992}. We may be able to picture how an individual plant learns to produce an individual good. But how does the organization learn when there are more variables at play? Learning must be taking a more complex form, when even the question of whether a goal was met is subject to a group’s sensemaking process, with subcoalitions at the table.
	
	When it is not clear what constitutes a reasonable performance, Cyert and March explain, organizations do not “maximize”, but turn to aspiration level \citep[p. 32]{Cyert1992}. In their original work, Cyert and March explain that this aspiration level is a function of current, or recent achievement levels, relative to a reference or peer group (\citealp[p. 39]{Cyert1992}; \citealp{Greve1998}; \citealp{Baum2007}).


\bibliography{bibliography}

\end{document}