\section{Conclusion}

The identification of organizational misconduct by social control agents does not cover all cases that are currently part of the public discourse. We offer an approach that would allow for accessing more publicly available data on misconduct. To move away from social control agents also deprives us of a very strong mechanism for elimination of this misconduct. Social control agents such as the government or the legal system have power over organizations, in a fashion that other audiences do not. Therefore, new mechanisms are needed for explaining how scandals based on misconduct play out. In particular, I suggest that the attention-based view of the firm could explain why scandals are effective in influencing organizational behavior, even in the absence of actions by social control agents. I further want to suggest that in the future, this mechanism could also explain why certain by social control agents seem to dissipate, while others make a big impact, with the support of the affected organizations themselves. One example is the framing of near-misses as successes, as we have also discussed in this article.