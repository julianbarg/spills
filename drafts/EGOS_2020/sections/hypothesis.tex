\section{Pipeline Spills in the US}

The US has constructed of the most extensive pipeline networks in the world \footnote{\url{See https://www.cia.gov/library/publications/the-world-factbook/fields/383.html, accessed 2020-01-13}}. In the decade from 2010 to 2020, US pipelines experienced y pipeline spills, z of which were categorized by the Pipeline and Hazardous Materials Safety Administration (PHMSA) as significant. Major operators regularly experience multiple significant spills per year.\footnote{See \url{https://julianbarg.shinyapps.io/incident_dashboard}, accessed 2019-01-13.} Yet, there is a significant variation in the number of pipeline incidents by operator, even after taking into account the number of pipeline miles or the amount of material transported.\footnote{For the full dataset, see \url{https://github.com/julianbarg/oildata/}} Even though pipelines are not typically classified as high reliability organizations such as nuclear reactors \citep[HROs;][]{Weick1999}, there is a complex system in place to collect data on pipeline spills, presumably to allow for control and learning.

PHMSA differentiates between three different classes of incidents. Serious incidents resulted either in a fatality or in-patient hospitalization. Note that thus a serious pipeline incident could hypothetically occur without any liquid being spilled. Regardless, I will henceforth use the term "spill" instead of the (more precise) euphemism "incident" that is used in the industry. Every serious spill is also automatically classified as a significant spill. As of 2002, a spill qualifies as "significant" also if either (2) \$50.000 or more of material or (3) 5 barrels or more were lost (50 barrels or more for high volatility liquids such as ethane, propane, or butane), or (4) if a fire or explosion occurred \footnote{See \url{https://www.phmsa.dot.gov/sites/phmsa.dot.gov/files/docs/pdmpublic_incident_page_allrpt.pdf}, accessed 2019-01-13.}. We will refer to spills below these thresholds as non-significant.

Social control agents have been identified in the literature on organizational misconduct as key actors that translate misconduct into scandals \citep{Greve2010}. However, the classification provided by social control agents is not always fine-grained or well-tuned. For instance, multinational corporations make sure first and foremost that their intricate tax evasion schemes are legal;\footnote{\url{https://www.bbc.com/news/magazine-20560359}, accessed 2019-01-13.} yet, a public discourse has emerged that without doubt would allow for researchers to explore this phenomenon under the umbrella of organizational misconduct. To return to the phenomenon at hand: over the decade 2010 to 2020, on average at least one significant spill occurred every three days. Therefore, the occurrence of a significant spill by itself would not be picked up by national media or an industry publication as a newsworthy event. Rather, an observer (or the regulator) would have to make some additional effort to either (1) discover an unusual pattern of spills, or to identify (2) unusual attributes (e.g., cause or magnitude) of an individual spill. In the classification system of the regulating agency, PHMSA, these spills would be no different than the significant spill which occur regularly, and, to date, did not prompt PHMSA to bring the hammer down. This occurred both (1) when in the case of Catholic Church, international media classified a series of sex abuses as a sex abuse scandal \citep{Piazza2018}, and (2) when in the case of stock-option backdating, auditors as a population level actors both disseminated and proactively eradicated the practice in their network \citep{}. We test two of these potential relationships between attributes of potential misconduct, and identification thereof as organizational misconduct.

\textbf{Hypothesis 1a}: \textit{In the absence of a clear assessment from a social control agents, high frequency of potential misconduct increases the likelihood of a general audience identifying a case of organizational misconduct.}

\textbf{Hypothesis 1b}: {\textit{In the absence of a clear assessment from a social control agents, extreme magnitude of potential misconduct increases the likelihood of a general audience identifying a case of organizational misconduct.}

\textbf{Hypothesis 1} (generalized): {\textit{In the absence of a clear assessment from a social control agents, unusual attributes of potential misconduct increases the likelihood of a general audience or population-level actor identifying a case of organizational misconduct.}