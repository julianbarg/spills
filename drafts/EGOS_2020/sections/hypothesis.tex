\section{General Audiences' Assessment of Oil Spills}

The US has constructed of the most extensive pipeline networks in the world \footnote{\url{See https://www.cia.gov/library/publications/the-world-factbook/fields/383.html, accessed 2020-01-13}}. In the decade from 2010 to 2020, US pipelines experienced y pipeline spills, z of which were categorized by the Pipeline and Hazardous Materials Safety Administration (PHMSA) as significant. Major operators regularly experience multiple significant spills per year.\footnote{See \url{https://julianbarg.shinyapps.io/incident_dashboard}, accessed 2019-01-13.} Yet, there is a significant variation in the number of pipeline incidents by operator, even after taking into account the number of pipeline miles or the amount of material transported.\footnote{For the full dataset, see \url{https://github.com/julianbarg/oildata/}} Even though pipelines are not typically classified as high reliability organizations such as nuclear reactors \citep[HROs;][]{Weick1999}, there is a complex system in place to collect data on pipeline spills, presumably to allow for control and learning.

PHMSA differentiates between three different classes of incidents. Serious incidents resulted either in a fatality or in-patient hospitalization. Note that thus a serious pipeline incident could hypothetically occur without any liquid being spilled. Regardless, I will henceforth use the term "spill" instead of the (more precise) euphemism "incident" that is used throughout the industry. Every serious spill is also automatically classified as a significant spill. As of 2002, a spill qualifies as "significant" also if either (2) \$50.000 or more of material or (3) 5 barrels or more were lost (50 barrels or more for high volatility liquids such as ethane, propane, or butane), or (4) if a fire or explosion occurred \footnote{See \url{https://www.phmsa.dot.gov/sites/phmsa.dot.gov/files/docs/pdmpublic_incident_page_allrpt.pdf}, accessed 2019-01-13.}. We will refer to spills below these thresholds as non-significant.

Social control agents have been identified in the literature on organizational misconduct as key actors that translate misconduct into scandals \citep{Greve2010}. However, the classification provided by social control agents is not always fine-grained or well-tuned. For instance, multinational corporations make sure first and foremost that their intricate tax evasion schemes are legal;\footnote{\url{https://www.bbc.com/news/magazine-20560359}, accessed 2019-01-13.} yet, a public discourse has emerged that without doubt would allow for researchers to explore this phenomenon under the umbrella of organizational misconduct. To return to the phenomenon at hand: over the decade from 2010 to 2020, on average at least one significant spill occurred every three days. Therefore, the occurrence of a significant spill by itself would not be picked up by national media or an industry publication as a newsworthy event. However, an observer (or the regulator) could exert some additional effort and either (1) discover an unusual pattern of spills, or to identify (2) unusual attributes (e.g., cause or magnitude) of an individual spill. In the classification system of the regulating agency, PHMSA, these spills would be no different than the significant spills which occur on a regular basis, and, as of 2020, have not prompted PHMSA to bring the hammer down. The identification by other audience occurred both (1) when in the case of Catholic Church, international media classified a series of sex abuses as a sex abuse scandal \citep{Piazza2018}, and (2) when in the case of stock-option backdating, auditors as industry-level actors both disseminated and proactively eradicated the practice in their network \citep{Mohliver2019}. In this article, we test the impact of the identification of misconduct by other audiences on the adaptation of pipeline safety by pipeline operators.
	
\subsection{Identification of misconduct and effect on an industry}

Typically, when a social control agent (especially the government) identifies a case of misconduct, the actor also has the power to end the practice. On the one hand, general audiences and industry-level actors do not hold the same power over organizations. On the other hand, organizations do themselves pay attention to the external environment. For example, both \citet{Kim2007} and \citet{Madsen2010} demonstrate that organizations respond to the failure and near-failure of other, similar organizations in their proximity. In extreme cases, a case of corporate misconduct can threaten the survival of an organization, which is a strong motivation for individuals in organizations to attent cases of misconduct that have been identified in their industry.

There are two potential mechanisms that would allow the monitoring of the environment to influence organizational behavior, when a general audience or industry-level actor identifies misconduct. (1) The bad reputation that results from organizational misconduct can hamper organizational performance \citep{Park2019}. It is beneficial for organizational survival to have mechanisms in place to respond to reputation risks. A possible response is to make efforts to improve pipeline safety, with the caveat that these are of limited effectiveness if the organization also engages in greenwashing to manage reputational risks \citep{Kim2015, Lyon2015}.

(2) The moral sentiment of the individual is another mechanism which influences organizational behavior. Because the phenomenon at hand is organizational misconduct, the research question is typically why individuals in organizations make unethical decisions \citep[e.g.,][]{Moore2008}. hence, the (uncontroversial) working assumption is that individuals in general can distinguish ethical and unethical behavior, and have an inclination to behave ethically. Cognitive dissonance theory holds that individuals seek to minimize the dissonance between their values and their actions \citep{Festinger1962}. Individuals can rationalize organizatoinal misconduct, e.g., through denying their responsibility or the occurance of an injury to a person \citep{Ashforth2003}. But in the face of organizational misconduct that unfolds as a scandal, it becomes more difficult to minimize the dissonance, and action become more likely.

\textbf{Hypothesis 1}: \textit{The identification of organizational misconduct by a general audience or industry-level actor promotes elimination thereof.}

Although public pressure is an important mechanism for organizational change, the ability of outside audiences to influence organizational behavior can be impaired where the organization and the outside audience do not see eye to eye. The opening paragraph of this article introduced an example where the outside audience viewed an operation as a certain failure, but a lobbyist interpreted the (prediced) outcome as a success. Failures of any magnitude provide an excellent opportunity for improvements, when they are taken serious, see e.g., high reliability organizations, where major failure events have to be absolutely ruled out \citep{Carroll1998}.

However, critical events are not always correctly recognized as such by the involved organizations. \citep{Madsen2010} points out that the space shuttles' loss of foam isolation preceding the *Columbia disaster* were not correctly interpreted as failure events, because the missions these shuttle launches were a part of overall were successful. In these preceding launches, a fatal accident was an outcome that could have occured, but by chance no such incident occurred before the \citep{Madsen2010}. In other words, managers made decisions that can be described as wrong, but these decisions were interpreted as correct because of the overall outcome \citep{Dillon2008}. Similarly, we would expect the failure of an organization to acknowledge a case of organiztational misconduct to stand in the way of the misconduct's elimination.

\textbf{Hypothesis 2}: \textit{The identification of near-misses as success hinders elimination of an identified organizational misconduct.}

The literature on organizational logics indicates that a critical piece, and a more general explanation to this process of recognition and elimination of organizational misconduct is the use of appropriate organizational frames \citep{Misangyi2008}. In the case of pipeline spills, the frame that could lead to an elimination or reduction of pipeline spills is that provided by a general audience or industry-level actor when a spill or spills are identified as organizatinal misconduct and translated into a scandal. An apotion of that outside frame could counteract the frame of near-misses as succes and its hinder hintering effect on improving pipeline safety.

\textbf{Hypothesis 3}: \textit{An organization's adaption of a frame from a general audience or industry-level actor promotes elimination of corporate misconduct.}