\section{}

The planning and construction of the Dakota Access Pipeline (DAPL) was accompanied by persistent protests. The indigenous protesters call themselves water protectors: in their epistemology, the pipeline represents a certain deterioration of their water supply.\footnote{See for example \url{https://www.businessinsider.com/north-dakota-access-pipeline-protest-drinking-water-2016-10}, accessed 2020-01-12.} The pipeline became operational in June 2017. Five spills occurred in 2017, and its sister pipeline, ETCO, experienced three. One of the ETCO spills was categorized by the Pipeline and Hazardous Materials Safety Administration (PHMSA) as "significant". These information did not fly under the radar, they received (albeit limited) coverage by the media.\footnote{\url{https://theintercept.com/2018/01/09/dakota-access-pipeline-leak-energy-transfer-partners/}, accessed 2019-01-13} Yet, high-ranking lobbyist and previous head of PHMSA Brigham McCown describes the pipeline's safety record as impressive and categorically rejects the notion that it could be a risk to public health.\footnote{\url{https://www.forbes.com/sites/brighammccown/2018/06/04/what-ever-happened-to-the-dakota-access-pipeline/}, accessed 2019-01-13.} How can these contradictory opposite frames exist side by side, and what is the implication of their coexistence for organizational behavior?

The existence of two radically diverging epistemologies on pipeline safety indicates two potential extensions of the work on organizational or corporate misconduct. (1) The divergence showcases the role of other audiences in addition to social control agents \citep{Greve2010} for the translation of misbehaviors into scandals. (2) The concept of near-misses that allows for the diverging interpretations \citep{Carroll1998, Dillon2008} has explanatory value in a more general model of organizational misconduct. Altogether, the divergence highlights the broader relevance of research on organizational misconduct. \citet{Mohliver2019} demonstrates that the identification of a practice as organizational misconduct can trigger both the adoption or cessation of this practice in an organization. But where the classification by a social control agent does not exist, or does not take the form of a binary label, the assessment can still steer organizations' behaviors. A positive or indifferent assessment can hinder organizations' aspirations to adapt, and thus indirectly supports existing routines. A negative or ambiguous assessment can trigger aspirations to adapt by abolishing a practice or reducing negative side effects thereof (e.g., oil spills).

Social control agents usually play a key role in the identifying misconduct \citep{Greve2010, Palmer2008, Schnatterly2018}. A government fine or the expulsion from a professional association is an unmistakable signal that individual or organizational misconduct has occurred. To use of decisions by social control agents is an advantageous method for the operationalization of misconduct; the researcher avoids the social dilemma of having to decide herself what constitutes an instance of misconduct \citep[e.g.,][]{Pontikes2010}. \citet{Mohliver2019} demonstrates that when a social control agent provides an indication that a practice constitutes misconduct, population-level or industry-level actors \citep{Madsen2018} such as professional auditors disseminate this practice throughout the population of organizations. The general public, especially media, can also play a key role in translating misconduct into scandals \citet{Piazza2018, Hoffman1999}. These two audiences act as alternative mechanisms in the identification of misconduct, in addition to their role in the unfolding of scandals.

Another new avenue is the organizational response, the interpretation of scandals as events by organizations. The translation of environmental feedback to organizational behavior into organizational adaption \citep{Cyert1963} does not occur mechanically; instead, it is mediated by the attentional processes of an organization \citep{Hoffman2001, Ocasio1997}. At the most basic level, a successful outcome reinforces existing routines, while failure triggers adaption of routines \citep{Levitt1988}. Thus, failures or scandals are particularly valuable for correcting organizational misdevelopments \citep{March1992}. This relationship is mediated by the attentional processes of the organization: where a failure goes unnoticed, for instance because it is mistakenly interpreted as a success, a potentially valuable opportunity for adaptation \citep{Carroll1998} is forgone \citep{Dillon2008}.

The translation of misconduct by a general audience such as the media into a scandal and its impact on organizations is only the first step. The attention of an organization is structurally determined \citep{Hoffman2001}, meaning that initial assessments by population-level actors are of particular importance for the proliferation or adaption of liminal or established practices \citep{Mohliver2019, Madsen2018}. Here, research on organizational misconduct could make a strong impact, by showing how frames e.g., provided by an industry-level actors, direct attention and allow practices to continue (or cease) in the absence of clear assessment by social control agents, even in the advent of resistance by general audiences. Or, organizations can adopt a frame from a general audience or industry-level actor that initiates adaption.