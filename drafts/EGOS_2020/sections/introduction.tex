\section{}

The planning and construction of the Dakota Access Pipeline (DAPL) was accompanied by persistent protests. The indigenous protesters call themselves water protectors; in their epistemology, the pipeline represents a certain deterioration of their water supply.\footnote{See for example \url{https://www.businessinsider.com/north-dakota-access-pipeline-protest-drinking-water-2016-10}, accessed 2020-01-12.} The pipeline became operational in June 2017. Five spills occurred in 2017, and its sister pipeline, ETCO, experienced three more. One of the ETCO spills was categorized by the Pipeline and Hazardous Materials Safety Administration (PHMSA) as "significant". These information did not fly under the radar, but were reported by the media.\footnote{\url{https://theintercept.com/2018/01/09/dakota-access-pipeline-leak-energy-transfer-partners/}} Yet, high-ranking lobbyist and previous head of PHMSA Brigham A. McCown describes the pipeline's safety record as impressive and categorically rejects the notion that it could be a risk to public health.\footnote{\url{https://www.forbes.com/sites/brighammccown/2018/06/04/what-ever-happened-to-the-dakota-access-pipeline/}} This article discusses both reasons for the coexistence of these polar opposite frames, and their implications for organizational behavior.

The existence of two radically diverging epistemologies on pipeline safety indicates two potential extensions of the work on organizational or corporate misconduct. (1) The divergence showcases the role of other audiences in addition to social control agents \citep{Greve2010} for the translation of misbehaviors into scandals. (2) Further, the  \citep{Carroll1998, Dillon2008} has explanatory value in a model of organizational misconduct.  In addition, the divergence highlights the broader relevance of research on organizational misconduct. \citet{Mohliver2019} demonstrates that the social assessment of a practice can trigger both the adoption or cessation of this practice in an organization. But where the classification by a social control agent does not exist, or does not take the form of a binary label, the assessment can still steer organizations' behaviors. A positive or indifferent assessment can hinder organizations' aspirations to take action on a practice or phenomenon, and thus indirectly supports it. A negative or ambiguous assessment can trigger aspirations to abolish a practice or reduce negative side effects thereof (e.g., oil spills).

Social control agents play a key role in the identifying misconduct \citep{Greve2010, Palmer2008, Schnatterly2018}. A government fine or the expulsion from a professional association is an unmistakable signal that individual or organizational misconduct has occurred. To use decisions by social control agents is a convenient method for the operationalization of misconduct, and avoids social dilemmas that would occur if the researcher were to decide herself or himself what constitutes an instance of misconduct \citep[e.g.,][]{Pontikes2010}. \citet{Mohliver2019} demonstrates that when a social control agent provides an indication that a practice constitutes organizational misconduct, population-level actors \citet{Madsen2018} such as professional auditors disseminate this practice throughout the population of organizations. The general public, especially media, can also play a key role in translating misconduct into scandals \citet{Piazza2018, Hoffman1999}.

In addition to the process of translating practices into scandals, the interpretation of scandals as events by organizations opens another avenue for inquiry. The translation of environmental feedback to organizational behavior into organizational adaption \citep{Cyert1963} does not occur mechanically; instead, it is mediated by the attentional processes of an organization \citep{Hoffman2001, Ocasio1997}. At the most basic level, a successful outcome reinforces existing routines, while failure triggers adaption of routines \citep{Levitt1988}. Thus, failures are particularly valuable for correcting organizational misdevelopments \citep{March1992}. This relationship is mediated by the attentional processes of the organization: where a failure goes unnoticed, for instance because it is mistakenly interpreted as a success, an opportunity for learning is forgone \citep[p. 457]{Madsen2010}.



In theory, near-misses provide more opportunities for learning (literature), but ...  forgone.
The opportunity to learn about... by near-misses


e.g., when a phenomenon only in retrospect reveals a systematic, underlying case of organizational misconduct \citet{Piazza2018}.