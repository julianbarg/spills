\documentclass[12pt, man, natbib]{apa6}
\usepackage[USenglish]{babel}
\usepackage{setspace}
\usepackage{hyperref}

\title{Pipelines, tension and Foucault}
\shorttitle{Pipelines, tension and Foucault}
\author{Julian Barg\\barg.julian@gmail.com}
\affiliation{Ivey Business School}
\setcitestyle{authoryear, open={()},close={)},citesep={,},aysep=}


% \abstract{}


\begin{document}
	
	\maketitle
	
	\singlespacing
	
	\section{}	

	Oil pipelines have been around since the 19th century. In the US, it is not unusual to see pipelines that have been under continuous operation since the 1930s. How safe are these pipelines? 
	
	If we were to take an optimistic view, we might think that since the pipeline industry now had multiple decades in a relatively stable environment, they would by now have mastered their craft. By precisely measuring the flow of the commodity through the pipeline at different points, leakages can be identified. And through preventive maintenance, cracks could be identified before they become a problem. Certainly, with the help of modern technology (which was not available when the brunt of pipelines were constructed over half a century ago) we could successfully tackle pipeline safety. More generally, this view encompasses that we get better at the tasks that we undertake, as knowledge is cumulative.
	
	At the same time, an overwhelming negative sentiment exists in our field. It seems that negative news about corporations never surprise (anybody?) anymore. Who would be surprised if we learned that pipeline operators largely have been operating the same pipelines since the 1970s, sometimes the 1930s, without any major upgrades, on a shoestring budget that barely allows for fixing things when they finally break. To appease the general public, they came up with the half-truth of technology that allows for prevent cracks from appearing and for detecting leaks.\footnote{Insert note on leaks below 1.5\% of flow going undetected for multiple days by plan for Keytstone pipeline}. When in reality, instead of just patching up a crumbling pipeline infrastructure, a major overhaul of the network would be due.
	
	A different debate has been taking place in the literature on organizational learning. The definition of organizational learning as "a change in the organization's knowledge that occurs as a function of experience" \citep[31]{Argote2013b} leaves some room for interpretation. The definition constitutes an elegant way to unite the two different roots of the learning literature. The first possible interpretation would be that knowledge is supposed to be cumulative. The word "function" then describes a mathematical function, as Argote does in the first chapter of her book.\footnote{Specifically, $y_i = ax^b_i$ \citep[11]{Argote2013a}, where i is the time subscript, y is the number of labor hours required to produce one unit of output, a is the number of hours required to produce the first unit of output, x is the number of units produced through time period i and b is the learning rate.}.

	Another interpretation [go to march, more qualitative]
	
	And finally, one could even debate the nature of experience. Does an organization yield new experience, if the events that occur do not fundamentally differ from previous events?

\bibliography{bibliography}

\end{document}