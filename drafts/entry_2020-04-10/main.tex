\documentclass[12pt, man, natbib]{apa6}
\usepackage[USenglish]{babel}
\usepackage{setspace}

\title{Pipelines, tension and Foucault}
\shorttitle{Pipelines, tension and Foucault}
\author{Julian Barg\\barg.julian@gmail.com}
\affiliation{Ivey Business School}
\setcitestyle{authoryear, open={()},close={)},citesep={,},aysep=}


% \abstract{}


\begin{document}
	
	\maketitle
	
	\singlespacing
	
	\section{}	

	Oil pipelines have been around since the 19th century. In the US, it is not unusual to see pipelines that have been under continuous operation since the 1930s. How safe are these pipelines? 
	
	If we were to take an optimistic view, we might think that since the pipeline industry now had multiple decades under relatively stable circumstances, they would by now have mastered their craft. By precisely measuring the flow of the commodity through the pipeline at different points, leakages can be identified. And by through preventive maintenance, cracks could be identified before they become a problem. Certainly, with the help of modern technology, that was not available when the brunt of the pipelines were constructed over half a century ago, we could successfully tackle pipeline safety. More generally, this view encompasses that we get better at the tasks that we undertake, as knowledge is cumulative.
	
	At the same time, an overwhelming negative sentiment exists in our field. It seems, negative news about a corporations never surprise (anybody?). Who would be surprised if we learned that pipeline operators largely have been operating the same pipelines since the 1970s, sometimes the 1930s, without any major upgrades, on a shoestring budget that barely allows for fixing things when they finally break. To appease the general public, they came up with the half-truth of technology that allows for prevent cracks from appearing and for detecting leaks \footnote{Insert note on keystone not quickly detecting leaks of <1.5\%.}.

\bibliography{bibliography}

\end{document}