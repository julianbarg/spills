\documentclass[12pt, man, natbib]{apa6}
\usepackage[USenglish]{babel}
\usepackage{setspace}
\usepackage{hyperref}

\title{Pipelines, tension and Foucault}
\shorttitle{Pipelines, tension and Foucault}
\author{Julian Barg\\barg.julian@gmail.com}
\affiliation{Ivey Business School}
\setcitestyle{authoryear, open={(},close={)},citesep={,},aysep=}


% \abstract{}


\begin{document}
	
	\maketitle
	
	\singlespacing
	
	\section{}	

	Oil pipelines have been around since the 19th century. In the US, it is not unusual to see pipelines that have been under continuous operation since the 1930s. How safe are these pipelines? 
	
	If we were to take an optimistic view, we might think that since the pipeline industry now had multiple decades in a relatively stable environment, they would by now have mastered their craft. By precisely measuring the flow of the commodity through the pipeline at different points, leakages can be identified. And through preventive maintenance, cracks could be identified before they become a problem. Certainly, with the help of modern technology (which was not available when the brunt of pipelines were constructed over half a century ago) we could successfully tackle pipeline safety. More generally, this view encompasses that we get better at the tasks that we undertake, as knowledge is cumulative.
	
	At the same time, an overwhelming negative sentiment exists in our field. It seems that negative news about corporations never surprise (anybody?) anymore. Who would be surprised if we learned that pipeline operators largely have been operating the same pipelines since the 1970s, sometimes the 1930s, without any major upgrades, on a shoestring budget that barely allows for fixing things when they finally break. To appease the general public, actors then come up with the half-truth of technology that allows for prevent cracks from appearing and for detecting leaks.\footnote{Insert note on leaks below 1.5\% of flow going undetected for multiple days by plan for Keytstone pipeline}. When in reality, instead of just patching up a crumbling pipeline infrastructure, a major overhaul of the network would be due.
	
	\subsection{The semantics of learning}
	
	A different but somewhat related debate has been taking place in the literature on organizational learning. Organizational learning is generally defined as "a change in the organization's knowledge [or behavior] that occurs as a function of experience" \citep[31]{Argote2013b}. This definition leaves some room for interpretation. The definition does constitute an elegant way to unite the two different roots of the learning literature. The first possible interpretation would be that knowledge is supposed to be cumulative. The phrase \textit{function of experience} then describes a mathematical function, as Argote does in the first chapter of her book.
	\footnote{Specifically, $y_i = ax^b_i$ where i is the time subscript, y is the number of labor hours required to produce one unit of output, a is the number of hours required to produce the first unit of output, x is the number of units produced through time period i and b is the learning rate.\citep[11]{Argote2013a}}.
	
	A version of learning that is developed by \citet{Levinthal1993} has a slightly different outlook on experience. In this version of learning, organizations do learn from experience, but it is the patterns they extrapolate that matter: when an organization yields positive results (relative to their aspiration), the organization interprets that as a confirmation of the current decision making process. In other words, success reinforces the current state of the organization. In this context, the change to knowledge as a \textit{function of experience} describes a different mechanism. \citet{March1991} for instance describe how as a function of an unusual experience, organizations can learn a lot. Learning is seen as a path-breaking change to an organization's understanding of the world. For example, a single landing that results in a fatal accident might yield many more insights for an airline than thousands of successful starts and landings.
	
	\subsection{The bigger picture}
	
	Which approach describes the real state of the world most accurately? Is an optimistic view of technology undue? Can organizations drive continuous improvement? To take an absolute position on either question seems awfully fatalistic. Instead, we could explore the limits toward either ends of the two continuums. 
	
	The oil industry, depending on the socio-political environment, can exert an awful lot of influence (directly) on the legislature and (indirectly through the media) on the general population. This influence can counteract a lot of the pressure that can result from negative impacts on stakeholders. On the other hand, the oil industry generally understands itself as one that is driven by (albeit sometimes primitive) technology, and it is dependent on insights and graduates in geology and engineering. It would be incorrect to say that there is a general ignorance or even animosity toward the sciences.
	
	What is the distinction between the insights that reach the industry, and insights that the industry insulates itself from, for instance in human geography or biology? For instance, in 2019 Eric Taylor, previous head of the Committee on the Status of Endangered Wildlife in Canada (CESEWIC) and professor of Zoology at UBC warned that any catastrophic event related to the Trans Mountain pipeline expansion "can completely wipe [the Steelhead Trout] out"\footnote{https://www.nationalobserver.com/2019/04/01/news/canada-considers-emergency-warning-scientists-could-complicate-trans-mountain}, based on his own data collection, and the quantitative, peer-reviewed work in his field \citep{Neilson2018}. Why isn't the knowledge that is created by him and his peers embraced by the industry?
	
	One could pose that question in terms of networks, but that would only move the discussion to another level without answering the underlying question. Why are broad bearers of a broad array of knowledge not in the network of the oil industry? The answer becomes obvious when we turn the question on its head. Why would the professor of zoology be in the network? There are discontinuities in the industries that have forced it to accept knowledge of "foreign" origin. In 1994, hurricane Rosa ravaged multiple pipelines that cross under a river in Houston. The extruding oil ignited, sending smoke several hundred feet into the air. Over 500 residents went to hospitals, mostly with inhalation injuries. Following that disaster, the industry began to procure academic knowledge on flood plains and flood planning.\footnote{The significance of this event should not be underestimated, as for most of the 20th century, the problem of floods was assumed to have been "solved" through dikes and systems of storm drains.}
	
	That is of course a far cry from the kind of knowledge that would suggest that pipelines are (at least in some localities) an unacceptable risk.

\bibliography{bibliography}

\end{document}