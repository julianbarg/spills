\documentclass[12pt, man, natbib]{apa6}
\usepackage[USenglish]{babel}
\usepackage{setspace}
\usepackage{hyperref}

\urlstyle{same}

\title{Operational Levels of Organizational Learning: From the Decision Maker to her or his Reference Group}
\shorttitle{Operational Levels of Organizational Learning}
\author{Julian Barg\\barg.julian@gmail.com}
\affiliation{Ivey Business School}
\setcitestyle{authoryear, open={()},close={)},citesep={,},aysep=}


% \abstract{}


\begin{document}

\maketitle

\singlespacing

\section{}	
	Decision making is central to the Carnegie School \citep{Gavetti2012}. \citet{Cyert1963} postulated that organizations do not maximize. Instead, organizations' behavioral patterns are the outcome of common characteristics of organizations \citep[e.g., the existence of routines or performance programs, see][]{March1958}, and bounded rationality of decision makers \citep{Gavetti2012}. This agenda for BTOF continues to exist until today \footnote{With some changes, such as \citet{Levitt1988}.}.
	
	How does this agenda of BTOF play out in the context of organizational learning? \citet{Levitt1988} provided the foundation for a theory of \textit{vicarious learning}; now \textit{vicarious learning} is present in most of the current literature on organizational learning. A contemporary view includes a sense of aspiration levels that trigger vicarious learning \citep{Baum2007}, and population-level processes which foster industry-wide learning \citep{Madsen2018}.
	
	We build on the original premise of the Carnegie School and extend on the current literature by analyzing the role of decision making in processes of organizational learning. Specifically, there are three relationships that deserve attention: (1) the relationship between population-wide processes, aspiration levels and learning, (2) the relationship between individuals' cognition and organizational learning (or a lack thereof), and (3) the relationship between departments and an organization's goals.
	
	Using a dataset of oil spills in the United States, we empirically study the three relationships. (1) Where it is not clear what constitutes a reasonable performance, social aspirations are at work. Learning processes then are a function of reference or peer groups (such as region and industry). When population-level actors (such as regulators or industry groups) clarify what constitutes a good performance, aspiration converges to a population level. As an empirical tool to test this, we leverage the 2002 redefinition of the PHMSA of what constitutes a significant pipeline incident, and the Association of Oil Pipe Lines' (AOPL) introduction of a zero incident goal.\footnote{\url{https://www.api.org/oil-and-natural-gas/wells-to-consumer/transporting-oil-natural-gas/pipeline/pipeline-safety}}
	
	(2) To bring the individual into the picture, we show a bias in performance toward round numbers \citet{Pope2011}. Where groups of individuals set goals (that are arbitrary in nature), they tend to choose round numbers, which is reflected in aspiration levels and learning outcomes. On the other hand, (3) individuals or departments have some arbitrage in reporting outcomes of learning, that they may use to set the future learning agenda. For instance, the PHMSA defines a significant incident as a liquid release of 50 barrels or more. The number of barrels released typically cannot be accurately measured. Thus, where an incident is in this neighborhood of 50 gallons, the individual who reports this incident can make a choice to either report a non-significant in the interest of meeting an organizational goal (when performance is in the neighborhood of aspiration); or the individual could report an incident when she or he feels that pipeline safety does not receive enough attention.
	
\bibliography{bibliography}

\end{document}